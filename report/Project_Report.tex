\documentclass[]{article}
\usepackage{lmodern}
\usepackage{amssymb,amsmath}
\usepackage{ifxetex,ifluatex}
\usepackage{fixltx2e} % provides \textsubscript
\ifnum 0\ifxetex 1\fi\ifluatex 1\fi=0 % if pdftex
  \usepackage[T1]{fontenc}
  \usepackage[utf8]{inputenc}
\else % if luatex or xelatex
  \ifxetex
    \usepackage{mathspec}
  \else
    \usepackage{fontspec}
  \fi
  \defaultfontfeatures{Ligatures=TeX,Scale=MatchLowercase}
\fi
% use upquote if available, for straight quotes in verbatim environments
\IfFileExists{upquote.sty}{\usepackage{upquote}}{}
% use microtype if available
\IfFileExists{microtype.sty}{%
\usepackage[]{microtype}
\UseMicrotypeSet[protrusion]{basicmath} % disable protrusion for tt fonts
}{}
\PassOptionsToPackage{hyphens}{url} % url is loaded by hyperref
\usepackage[unicode=true]{hyperref}
\hypersetup{
            pdfborder={0 0 0},
            breaklinks=true}
\urlstyle{same}  % don't use monospace font for urls
\usepackage[margin=1in]{geometry}
\usepackage{longtable,booktabs}
% Fix footnotes in tables (requires footnote package)
\IfFileExists{footnote.sty}{\usepackage{footnote}\makesavenoteenv{long table}}{}
\usepackage{graphicx,grffile}
\makeatletter
\def\maxwidth{\ifdim\Gin@nat@width>\linewidth\linewidth\else\Gin@nat@width\fi}
\def\maxheight{\ifdim\Gin@nat@height>\textheight\textheight\else\Gin@nat@height\fi}
\makeatother
% Scale images if necessary, so that they will not overflow the page
% margins by default, and it is still possible to overwrite the defaults
% using explicit options in \includegraphics[width, height, ...]{}
\setkeys{Gin}{width=\maxwidth,height=\maxheight,keepaspectratio}
\IfFileExists{parskip.sty}{%
\usepackage{parskip}
}{% else
\setlength{\parindent}{0pt}
\setlength{\parskip}{6pt plus 2pt minus 1pt}
}
\setlength{\emergencystretch}{3em}  % prevent overfull lines
\providecommand{\tightlist}{%
  \setlength{\itemsep}{0pt}\setlength{\parskip}{0pt}}
\setcounter{secnumdepth}{5}
% Redefines (sub)paragraphs to behave more like sections
\ifx\paragraph\undefined\else
\let\oldparagraph\paragraph
\renewcommand{\paragraph}[1]{\oldparagraph{#1}\mbox{}}
\fi
\ifx\subparagraph\undefined\else
\let\oldsubparagraph\subparagraph
\renewcommand{\subparagraph}[1]{\oldsubparagraph{#1}\mbox{}}
\fi

% set default figure placement to htbp
\makeatletter
\def\fps@figure{htbp}
\makeatother

\usepackage{float}
\usepackage{titling}
\usepackage{caption} 
\usepackage{amsmath}
\usepackage{booktabs}
\captionsetup[table]{skip=8pt}
\usepackage{booktabs}
\usepackage{longtable}
\usepackage{array}
\usepackage{multirow}
\usepackage{wrapfig}
\usepackage{float}
\usepackage{colortbl}
\usepackage{pdflscape}
\usepackage{tabu}
\usepackage{threeparttable}
\usepackage{threeparttablex}
\usepackage[normalem]{ulem}
\usepackage{makecell}
\usepackage{xcolor}

\author{}
\date{\vspace{-2.5em}}

\begin{document}

\begin{flushleft}
\LARGE{\textbf{Project Report: Measures and Indicators of Suicide Rates by Country}}\\
\vspace*{2\baselineskip}
\Large{Georgia Tech: ISyE 6414 - Dr. Yajun Mei}\\
\vspace*{3\baselineskip}
\Large{\textbf{Team Members}}\\
Samuel Garcia\\
Michael Szostak\\ 
Osman Ghandour\\ 
Peter Williams\\
\vspace*{2\baselineskip}
\Large{\textbf{Report Date}}\\
April 21, 2020
\newpage
\end{flushleft}

{
\setcounter{tocdepth}{2}
\tableofcontents
}
\newpage

\section{Abstract}\label{abstract}

\emph{informative summary of the whole report (100-500 words).}

\section{Introduction}\label{introduction}

\subsection{Problem Description and
Motivation}\label{problem-description-and-motivation}

According to the \emph{World Health Organization}, (``Suicide: One
Person Dies Every 40 Seconds'' 2019) one person dies every 40 seconds
from suicide. It is the second leading cause of death among teenagers
and adults aged 15-29 years. Despite the staggering number of suicides
happening worldwide, just 38 governments worldwide have a national
suicide prevention strategy.

Each of these deaths are tragic, and sadly also preventable. For every
suicide, there are many more attempts, and previous suicide attempts are
the single most important predictor or risk factor for future suicide
attempts. (``Suicide: Key Facts'' 2019) The possibility of prevention
and the scale of the problem highlight the need for policy makers, at
the national level, to understand the factors that contribute to suicide
not only in their own nations but also globally.

The objective of our project is to analyze country-level and gender
specific data related to suicide rates over the past ten years, and
augment it with other country level data and metadata to gain an
understanding and better describe rising suicide rates worldwide. Our
intentions are two-fold:

\begin{enumerate}
\def\labelenumi{\arabic{enumi}.}
\tightlist
\item
  To describe and identify measures and indicators that impact suicide
  rates at the country level for both males and females, in order to
  provide high-level decision making support for leaders and authors of
  public policy related to mental health and suicide. This includes
  identifying, and monitoring over time, meaningful factors and measures
  related to suicide prevention, for those who manage health related
  planning activities and priorities.\\
\item
  To discuss the sources, and process of collection, of data related to
  suicide at the country level; describe ancillary datasets that were
  employed to augment and enrich insights related to international
  differences in suicide rates. The goal is to make further research and
  data analysis on this topic more accessible to other interested
  researchers and data analysts in the future.
\end{enumerate}

\subsection{Challenges}\label{challenges}

\subsection{Problem Solving
Strategies}\label{problem-solving-strategies}

\subsection{Learnings}\label{learnings}

\subsection{Report Outline}\label{report-outline}

\section{Problem Statement}\label{problem-statement}

\emph{cite the data sources, and provide a simple presentation of data
to help readers understand the problem or challenge(s).}

\begin{enumerate}
\def\labelenumi{\arabic{enumi}.}
\item
  Can we statistically confirm and identify which measures/indicators
  are best associated with male and female suicide rates at the country
  level?
\item
  Based on the data we obtain, are there any inferences we can make
  about how a country may be able to reduce the suicide rate (male,
  female, or both)?
\item
  Are there recommendations we can make to future researchers or policy
  makers beginning to study this topic?
\end{enumerate}

\subsection{Data Sources}\label{data-sources}

The core dataset we plan to rely on comes directly from the \emph{World
Health Organization} and is referenced below. The key measure of
interest for our study is the \emph{the age-standardized suicide rate}
by country, which is defined as \emph{a weighted average of the
age-specific mortality rates per 100,000 persons, where the weights are
the proportions of persons in the corresponding age groups of the WHO
standard population.} (``Suicide Rate Estimates, Age-Standardized
Estimates by Country'' 2019) The data consists of country-level measures
of suicides rates, as defined above, for \(183\) nations, broken out by
gender, and standardized as describe above. These estimates of
age-standardized suicide rates were taken in the year \(2000\),
\(2010\), \(2015\), and \(2016\). This enables the data analyst to
observe differences in rate estimates over time, allowing more context
and benchmarking for the purposes of this analysis. In addition to
relying on the core suicide rate statistics provided above, we also
intend to append country-level data, for corresponding time periods,
from ancillary data sources. Currently, we are considering the following
data:

\begin{enumerate}
\def\labelenumi{\arabic{enumi}.}
\item
  GDP per capita, as a high level measure of wealth, sourced from the
  \emph{World Bank}
  (\url{https://data.worldbank.org/indicator/NY.GDP.PCAP.PP.CD?view=map})
\item
  Adult education level, as made available from the \emph{OECD}, the
  \emph{Organization for Economic Co-operation and Development} (
  \url{https://data.oecd.org/eduatt/adult-education-level.htm} )
\item
  The female labor force participation rate, from the \emph{United
  Nations Development Programme}
  (\url{http://hdr.undp.org/en/content/labour-force-participation-rate-female-male-ratio}
  )
\item
  Whether or not the government of a country has a suicide prevention
  strategy, according to the \emph{World Health Organization}
  (\url{https://apps.who.int/iris/rest/bitstreams/1174021/retrieve})
\end{enumerate}

\section{Proposed Methodology}\label{proposed-methodology}

\emph{explain (and justify) your proposed methods or models.}

\section{Analysis and Results}\label{analysis-and-results}

\emph{present key findings when executing the proposed methods or
models. For the benefit of readability, detailed results should be
placed in the Appendix. Reference of computer softwares to implement
your proposed methods or models (even it is a web page) should be
given.}

\section{Conclusions}\label{conclusions}

\section{Appendix}\label{appendix}

\emph{This section only includes needed documents to support the
presentation in the report. Feel free to divide it into several
subsections if necessary. Do NOT dump all computer outputs unor-ganized
here}

\emph{References and Credits}

\hypertarget{refs}{}
\hypertarget{ref-whodat2019}{}
``Suicide Rate Estimates, Age-Standardized Estimates by Country.'' 2019.
\url{http://apps.who.int/gho/data/node.main.MHSUICIDEASDR?lang=en}.

\hypertarget{ref-stats2019}{}
``Suicide: Key Facts.'' 2019.
\url{https://www.who.int/news-room/fact-sheets/detail/suicide}.

\hypertarget{ref-who2019}{}
``Suicide: One Person Dies Every 40 Seconds.'' 2019.
\url{https://www.who.int/news-room/detail/09-09-2019-suicide-one-person-dies-every-40-seconds}.

\end{document}
