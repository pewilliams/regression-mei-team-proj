\documentclass[]{article}
\usepackage{lmodern}
\usepackage{amssymb,amsmath}
\usepackage{ifxetex,ifluatex}
\usepackage{fixltx2e} % provides \textsubscript
\ifnum 0\ifxetex 1\fi\ifluatex 1\fi=0 % if pdftex
  \usepackage[T1]{fontenc}
  \usepackage[utf8]{inputenc}
\else % if luatex or xelatex
  \ifxetex
    \usepackage{mathspec}
  \else
    \usepackage{fontspec}
  \fi
  \defaultfontfeatures{Ligatures=TeX,Scale=MatchLowercase}
\fi
% use upquote if available, for straight quotes in verbatim environments
\IfFileExists{upquote.sty}{\usepackage{upquote}}{}
% use microtype if available
\IfFileExists{microtype.sty}{%
\usepackage[]{microtype}
\UseMicrotypeSet[protrusion]{basicmath} % disable protrusion for tt fonts
}{}
\PassOptionsToPackage{hyphens}{url} % url is loaded by hyperref
\usepackage[unicode=true]{hyperref}
\hypersetup{
            pdfborder={0 0 0},
            breaklinks=true}
\urlstyle{same}  % don't use monospace font for urls
\usepackage[margin=1in]{geometry}
\usepackage{graphicx,grffile}
\makeatletter
\def\maxwidth{\ifdim\Gin@nat@width>\linewidth\linewidth\else\Gin@nat@width\fi}
\def\maxheight{\ifdim\Gin@nat@height>\textheight\textheight\else\Gin@nat@height\fi}
\makeatother
% Scale images if necessary, so that they will not overflow the page
% margins by default, and it is still possible to overwrite the defaults
% using explicit options in \includegraphics[width, height, ...]{}
\setkeys{Gin}{width=\maxwidth,height=\maxheight,keepaspectratio}
\IfFileExists{parskip.sty}{%
\usepackage{parskip}
}{% else
\setlength{\parindent}{0pt}
\setlength{\parskip}{6pt plus 2pt minus 1pt}
}
\setlength{\emergencystretch}{3em}  % prevent overfull lines
\providecommand{\tightlist}{%
  \setlength{\itemsep}{0pt}\setlength{\parskip}{0pt}}
\setcounter{secnumdepth}{0}
% Redefines (sub)paragraphs to behave more like sections
\ifx\paragraph\undefined\else
\let\oldparagraph\paragraph
\renewcommand{\paragraph}[1]{\oldparagraph{#1}\mbox{}}
\fi
\ifx\subparagraph\undefined\else
\let\oldsubparagraph\subparagraph
\renewcommand{\subparagraph}[1]{\oldsubparagraph{#1}\mbox{}}
\fi

% set default figure placement to htbp
\makeatletter
\def\fps@figure{htbp}
\makeatother

\usepackage{float}
\usepackage{titling}
\usepackage{caption} 
\usepackage{amsmath}
\usepackage{booktabs}
\captionsetup[table]{skip=8pt}
\usepackage{abstract}
  \renewcommand{\abstractnamefont}{\normalfont\large\bfseries}
\usepackage{booktabs}
\usepackage{longtable}
\usepackage{array}
\usepackage{multirow}
\usepackage{wrapfig}
\usepackage{float}
\usepackage{colortbl}
\usepackage{pdflscape}
\usepackage{tabu}
\usepackage{threeparttable}
\usepackage{threeparttablex}
\usepackage[normalem]{ulem}
\usepackage{makecell}
\usepackage{xcolor}

\author{}
\date{\vspace{-2.5em}}

\begin{document}

\begin{flushleft}
\LARGE{\textbf{Country Level Indicators of Suicide Risk:\\ Data Analysis \& Decision Support for Policy Makers}}\\
\vspace*{2\baselineskip}
\Large{Project Report: Georgia Tech ISyE 6414 - Dr. Yajun Mei}\\
\vspace*{3\baselineskip}
\Large{\textbf{Team Members}}\\
Samuel Garcia\\
Michael Szostak\\ 
Osman Ghandour\\ 
Peter Williams\\
\vspace*{2\baselineskip}
\Large{\textbf{Report Date}}\\
April 21, 2020
\newpage
\end{flushleft}

{
\setcounter{tocdepth}{2}
\tableofcontents
}
\newpage

\begin{abstract}
\bigskip
Relying on open source data from the World Health Organization and other non-governmental bodies, we highlight trends and related to country-level measures of suicide globally. A descriptive model is formulated that identifies a set of meaningful factors and measures that highlights some intuitive relationships between country-level suicide rates and other indicators such as income, alcohol consumption, and the presence of a country level suicide prevention strategy.

An overview of our the sources of data utilized, and the process of collection of data related to suicide at the country level is described briefly. A description of some ancillary datasets that were employed to augment and enrich insights is also provided. The intention of these descriptions is to make further research and data analysis on this topic more accessible to other interested researchers and data analysts in the future.  

An initial framework for data-driven decision support for policy makers is presented with recommendations including the identification and ongoing monitoring of meaningful factors and measures related to suicide prevention. The hope is that this can provide some introductory guidance for those managing health related planning activities and priorities related to this topic.  

An initial framework for data-driven decision support for policy makers is presented with recommendations including the identification and ongoing monitoring of meaningful factors and measures related to suicide prevention. The hope is that this can provide some introductory guidance for those managing health related planning activities and priorities related to this topic.  

Utilizing insights from our data analysis, as well as secondary research sources, we present some recommendations related to health-planning related activities for policy makers. Limitations to our research are also discussed in context of the scope of the data collected and the methodology presented here.   


\end{abstract}

\section{Research Motivation}\label{research-motivation}

Suicide is a complex societal problem with multiple social,
psychological, biological, and cultural factors. It is one of the top 20
leading causes of death in the world for all ages (5). An estimated one
million people die annually from suicide, i.e., a global mortality rate
of 16 per 100,000, or one death every 40 seconds (5). Due to the
interactions of so many factors, suicide has no singular cause.

Though it might seem intuitive to categorize suicidal ideation,
attempted suicide, and completed suicide as strictly a psychiatric or
medical issue or a mental illness, not all who commit suicide are
mentally ill. ``Mental illness is often not clearly distinguishable from
normal distress'' (11). Stressful experiences, such as exposure to
trauma, the death of a loved one, a job loss, a change in physical
health or relationships and individual characteristics and behaviors are
also associated with suicide (13).

\subsection{Where Is This a Problem}\label{where-is-this-a-problem}

Suicide is the 15th leading cause of death worldwide, with over 75\% of
suicides occurring in low-income and middle-income countries (15).
Poverty, particularly in the form of worse economic status, diminished
wealth, and unemployment is associated with suicide (15). Poverty may be
defined in terms of deprivation across the multiple dimensions of life,
such as education, health, or housing (6). Both chronic poverty and
acute economic events, such as crop failure, constitute possible risk
factors for suicidal ideations and behaviors (15).

Poverty, unemployment, illiteracy, lack of civic facilities, poor access
to health facilities, the absence of health insurance or of welfare are
factors that adversely impact upon the overall mental health status of
the population (17). In developing countries, the interval between onset
of suicidal ideation and the act of suicide is frequently
overlooked---partly because of ignorance but also because families and
subjects do not know where to seek help. Even when they do realize
something is wrong, they lack resources to seek help. (17)

To underscore the complex nature of the suicide problem, and to show how
causes of suicide can vary between countries, we contrast the situations
in Zimbabwe and Russia. Zimbabwe has suffered endemic poverty,
hyperinflation, and high unemployment for years. On the other hand,
Russia's levels of alcohol consumption are among the highest in the
world. Though their underlying conditions appear to be markedly
different, both nations suffer from high rates of suicide.

\subsubsection{Economics - Zimbabwe}\label{economics---zimbabwe}

Endemic poverty, hyperinflation, and an unemployment rate of over 90\%
(12) are among the economic and social problems plaguing Zimbabwe, where
political crisis coupled with failed economic policy have led to its
decline. Zimbabwe's economic woes are often attributed to the policies
of former dictator Robert Mugabe. Post Mugabe, Zimbabwe continues to
deal with debt issues, difficulty attracting foreign investment, and
currency instability.

The WHO estimates that 19 persons per 100k take their own life
deliberately in Zimbabwe per annum (2019). Of the 166 countries in our
study, Zimbabwe ranks 13th in the world for suicides per capita.

\emph{Figure 1: Suicide Rate in Africa (annual persons per 100k
population)}

\begin{center}\includegraphics{Project_Report_files/figure-latex/africa_map_plot-1} \end{center}

\emph{Sourced from the World Health Organization report: ``Suicide: Key
Facts, 2019'' and the WorldBank Economic Profile of the Country of
Zimbabwe}

\subsubsection{Alcohol Abuse - Russia}\label{alcohol-abuse---russia}

In general, there is no single factor responsible for the suicide rate.
Globally however, harmful use of alcohol is among the major risk factors
for suicide.

A study published in The Lancet found that global alcohol consumption
saw an increase of about 70\% from 1990 to 2017, going from about 21
billion liters of pure alcohol to 35.7 billion liters of pure alcohol
(7). Countries that have higher rates of alcohol use generally also have
higher rates of suicide. Current evidence indicates an association
between alcohol dependence and impulsive suicide attempts (5).

Alcohol use disorder (AUD), defined in the WHO's International
Classification of Diseases, is a chronic disease characterized by
compulsive alcohol consumption, loss of control over of alcohol intake,
and negative emotional state when not consuming alcohol. Alcohol
intoxication can increase dysphoria, cognitive dysfunction, impulsivity
and suicidal ideation. People have approximately seven times increased
risk for a suicide attempt soon after drinking alcohol, and this risk
further increases to 37 times after heavy use of alcohol (4). Risk of
suicidal ideation, suicidal attempts and completed suicide are each
increased by 2--3 times among those with Alcohol Use Disorders (AUD) in
comparison with the general population (5).

In Russia, the prevalence of AUD is about 4.7\%, meaning that almost
1-in-20 suffer from alcohol dependence. Alcoholism has been a problem
because drinking is not only pervasive, but also a socially acceptable
behavior in Russian society. The WHO estimates that 27 persons per 100k
take their own life deliberately in Russia per annum (2019). Of the 166
countries in our study, Russia ranks 3rd in the world for suicides per
capita.

\emph{Figure 2: Suicide Rate in Russia (annual persons per 100k
population)}

\begin{center}\includegraphics{Project_Report_files/figure-latex/russia_map_plot-1} \end{center}

\emph{Sourced from the World Health Organization report: Suicide: Key
Facts, 2019}

\section{Variables \& Data Sources}\label{variables-data-sources}

Due to the sheer number of potential factors associated with suicide and
the complex nature of the relationships between them, we wanted to
identify those that were best associated with suicide rates at the
country level. We chose to limit our study to a small set of factors
that could be controlled for and acted upon via policy interventions.
The domains from which we drew the factors, had to be broad enough to
reasonably represent as many of the potential causes or mitigators of
suicide as possible.

Among the domains in consideration were lifestyle, medical/mental
health, economic, and suicide-focused policy. The core dataset we plan
to rely on comes directly from the World Health Organization
(\emph{WHO}). The key measure of interest for our study is the
age-standardized suicide rate by country, which is defined as a weighted
average of the age-specific mortality rates per 100,000 persons, where
the weights are the proportions of persons in the corresponding age
groups of the WHO standard population, see the Appendix section for
details on how this is estimated. Estimates of age-standardized suicide
rates were taken in the year with the most recent available data for
each country from the \emph{WHO}.

In addition to relying on the core suicide rate statistics provided
above, we intend to append country-level data from ancillary data
sources. Health Expenditure and GDP per capita were chosen to reflect
the resources that a country has its disposal to reduce the suicide
rate. Liters of Alcohol per capita was chosen to account for an aspect
of culture (alcohol consumption) that the media often links to mental
health outcomes. The prevalence of a suicide prevention strategy, the
number of psychiatrists, and the number of mental hospitals were chosen
to reflect how a country has deployed its resources to improve mental
health outcomes. The female/male labor participation ratio was included
to control for this aspect of a country's culture. The data for these
variables along with the suicide rate variable was available for 166
countries.

\section{Modeling \& Assumptions}\label{modeling-assumptions}

We developed a multiple linear regression model to infer properties
about how a handful of socioeconomic and cultural indicators impact
suicide rates. An important distinction is that the model is intended to
be used for inferential, rather than predictive purposes. Our objective
is to discover relationships between variables to inform relevant public
policy and future research in the area.

The first step in developing the model was to transform the outcome
variable. This transformation (Box-Cox) was used to make the outcome
variable `more normal', and it helped to characterize relationships
between variables in our data. The next step was to remove outliers. We
used regression diagnostics and visual data exploration to identify
unusual data points. Brief qualitative research was then conducted on
the country represented by each point to confirm whether or not the
point should be removed. After the removal of the outlier points, we
utilized a stepwise algorithm to identify which variables should be
included in our model. This `automatic' procedure yielded the set of
variables that we would analyze more closely.

The following variables were selected and included in our model: labor
force participation rate (female-male ratio), GDP per capita (PPP),
liters of alcohol consumption per capita, and the prevalence of a
national suicide prevention strategy. The algorithm excluded the
following variables from the model: current health expenditure as a
percentage of GDP, the number of psychiatrists working in the mental
health sector (per 100k pop.), and the number of mental hospitals (per
100k pop.).

The final step in the development of our model was to implement an
iterative algorithm that adjusted the weights for each of our data
points (Iteratively Reweighted Least Squares). This allowed us to
further limit the influence of outliers on our data.

In the development of our model, we relied on a few assumptions about
the quality of our data. The first is regarding GDP per capita, which is
assumed to be an appropriate indicator to reflect the wealth of a
country. The second relates to the prevalence of a national suicide
prevention strategy. It is assumed that the presence of such a strategy
is indicative that the country has taken the time to develop a
comprehensive and data driven approach to suicide, based on solid
evidence. We also assume that the liters of alcohol consumed per capita
reflects the tendency for individuals in the given country to consume
excessive amounts of alcohol.

\section{Quantifying Impact of Measures on
Suicide}\label{quantifying-impact-of-measures-on-suicide}

The model described allows the data analyst to describe the
relationships between country-level indicators and measures and suicide
rates globally. Armed with a descriptive model of suicide rates,
decision-makers can quantify the relationships between these measures to
support insight for their health related planning activities. However,
data-driven insights derived from this model and the data sources
highlighted in this report should be considered in context of specific
country-level impacts not considered in this report. As policy makers
infer correlations of country-level measures and indicators with suicide
rates there is a need to continue to engage subject-matter-experts in
the field to draw on their knowledge and experience. Our intention is to
provide some initial context and decision support for policy makers
managing health related planning globally and at the country level, but
the limitations of our research and methodology highlight the ongoing
need for data-driven insights to be utilized in context of other
research available, as well as the domain knowledge of practitioners,
health professionals, scientists, and policy makers among others.

To highlight the need for a holistic approach to gathering data-driven
insights, and incorporating domain knowledge of subject matter experts,
we describe a basic framework for potentially incorporating insights
from our model into a policy decision-making process:

\begin{table}[H]
\centering 
\caption{Framework: Identifying, Describing and Monitoring Country Level Indicators to Support Decision-Making}
\
\begin{tabular}{p{7cm}p{9cm}}  
\hline  
   Area of Focus  & Decision Support  \\   
\hline 
 Identifying \& Quantifying Measures and Indicators of Country Level Suicide Rates &  Using a model to describe the relationships between country-level indicators and suicide rates to help policy makers identify what variables are important and putting context around how to monitor them \\   
 \hline 
Incorporating Domain Knowledge and Expertise of Subject Matter Experts & Allows policy makers to correlate indicators with country-level suicide related outcomes in context of qualitative insights from experts in the field \\   
\hline 
Insight Gathering, Analysis and Support Policy Maker Decisions & Integrating data and insights and providing inital context to help policy makers to frame longer term health planning activities and policies \\
\hline 
\end{tabular} 
\end{table}

In context of providing exemplary decision-support ofr health related
planning activities we discuss some interesting relationships we
identified between country-level suicide rates and the descriptor
variables we selected in the following sections.

\subsection{Quantifying Impact: Income, GDP per
person}\label{quantifying-impact-income-gdp-per-person}

A key insight that surfaced during the modeling process was the presence
of a significant relationship between a measure of income (here defined
as the country-level GDP per person) and suicide country level suicide
rates. Countries with lower per person income, tend to have higher
incidence of suicide when controlling for other variables in our model.
To explore this relationship further, we wanted to explore the expected
impact of income, or per-person GDP, on a country's suicide rate
controlling for the other key factors in our model including alcohol
consumption, the female-male labor participation rate, and the presence
of a country-level suicide prevention strategy.

When controlling for these variables, we found that an approximate 10\%
increase in income (or GDP per-person) corresponded to a 2\% decrease in
suicide rate at the country level for the typical country. The
illustrate further the following plot highlights the sensitivity of
income to suicide rates based on results from our model:

\emph{Figure 3: Expected Country-Level Suicide Rate vs.~Income (GDP
per-person)}

\begin{center}\includegraphics{Project_Report_files/figure-latex/agdp_plot-1} \end{center}

\emph{Note: Expected rate with 95\% confidence intervals for the typical
country (i.e.~holding other variables in model identified fixed at the
`sample mean').}

\subsection{Quantifying Impact: Alcohol
Consumption}\label{quantifying-impact-alcohol-consumption}

A key insight that surfaced during the modeling process was the presence
of a significant relationship between a measure of alcohol abuse, liters
of consumption per year, and suicide country level suicide rates. As
might be expected, countries with higher levels of alcohol consumption
income in our data, tended to have higher incidence of suicide when
controlling for other variables in our model. Based on our estimates, an
approximate 4\% increase in alcohol consumption corresponded to an
expected 2\% increase in suicide rate at the country level for the
typical country, i.e.~countries with \textgreater{}4 liters of
consumption per adult, on average, per year.

Another thing to note from our data analysis, is that alcohol
consumption was the most impactful and significant indicator of
country-level suicide rate we identified. The illustrate the estimated
impact, the following plot highlights the sensitivity of alcohol
consumption to suicide rates based on results from our model:

\emph{Figure 4: Expected Country-Level Suicide Rate vs.~Liters of
Alcohol Consumer Per Year (pp)}

\begin{center}\includegraphics{Project_Report_files/figure-latex/a_alc_plot-1} \end{center}

\emph{Note: Expected rate with 95\% confidence intervals for the typical
country (i.e.~holding other variables in model identified fixed at the
`sample mean').}

This sensitivity analysis highlights the strong relationship between
these variables, which isn't an entirely novel relationship we
discovered. Our brief case study of suicide in Russia was meant to
provide some discussion of the real impact alcohol consumption. However
our model does provide estimates that can begin to estimate the impact
of this factor quantitatively at the country level.

\subsubsection{Quantifying Impact: The Presence of A National Suicide
Strategy}\label{quantifying-impact-the-presence-of-a-national-suicide-strategy}

A key variable we wanted to control for in our model was an indicator of
a country having a national suicide prevention strategy in place. Based
on data criteria from the \emph{WHO}, we measured the impact of having a
stand-alone national suicide prevention strategy. According to the
\emph{WHO} the categorical criterion for the presence of a national
suicide strategy was that, in order to be considered, as country's
planplan have been stand-alone, and not be integrated into another plan.

As we looked at the data, we found that the presence of a suicide
prevention strategy was more likely to be associated with countries
struggling with suicide prevention overall. Some of these countries
included Guyana, Lithuania, Suriname, Belarus and South Korea.

Based on our estimates from our model, countries that have implemented a
suicide prevention strategy have a 26\% higher incidence of suicide
nationally as shown below:

\emph{Figure 5: Expected Country-Level Suicide Rate vs.~The Presence of
A National Suicide Strategy}

\begin{center}\includegraphics{Project_Report_files/figure-latex/sstrat_plot-1} \end{center}

\emph{Note: Expected rate with 95\% confidence intervals for the typical
country (i.e.~holding other variables in model identified fixed at the
`sample mean').}

While in may seem unintuitive that countries with with a suicide
prevention strategy have higher rates, it should not be inferred or
interpreted from our model that having a national suicide prevention
strategy leads to incidence of suicide rates overall. It may be
understood that the institution of this strategy may be a `reactive'
decision, that is, the presence of a strategy has been instituted as a
result of high incidence. As a result, we wanted to ensure we were
incorporating this variable in our model as both a control for
estimating other effect, and for providing context for countries with
high rates of suicide otherwise.

\section{Recommendations and Decision
Support}\label{recommendations-and-decision-support}

For the selected inputs chosen in the model, there are corresponding
recommendations for each input. The following sections go over
recommendations for each model input:

\subsection{Suicide Prevention
Strategy}\label{suicide-prevention-strategy}

National Suicide prevention strategies have been implemented in many
countries to combat suicide. Many have found their own way of handling
the problem but there was not widespread acceptance and organizational
response to the problem until recently. In 1993 The United Nations
created a task force which teamed up with the WHO to put together a
study on the causes, preventative, and rehabilitative measures of
suicide, and which culminated with the release of a report in 1996
called ``Prevention of suicide: guidelines for the formulation and
implementation of national strategies'' (1). Before this Finland was the
only national government to which had a national program for suicide
prevention.

These guidelines were followed to varying degrees by different countries
or local municipalities. The 1996 study was followed up in 2018 with a
study called ``National suicide prevention strategies; progress,
examples, and indicators'' (2) which contained updated recommendations
and findings since 1996. For instance, the intersection of biological,
psychological, social, environmental, and cultural factors which
influence suicide, as well as successful policies which countries which
countries which had national suicide prevention programs had
implemented. It is from this 2018 study which contained a list of all
countries which ``stand-alone national suicide prevention strategy
(NSPSs) adopted by the government'' was drawn for our research.

We found that countries that have put a national suicide prevention
strategy in place, tend to have higher incidence of suicide rates
overall, this may be reactionary in nature. If a country has high
suicide rates then more attention is paid to the issue and a programs
are put in place to combat it, thus since most National suicide programs
have only been emplaced in the last two decades this may explain the
counterintuitive trend observed. However, it is still advised to have a
national strategy to address suicide.

Government policy to combat suicide allows for the ``development and
strengthening surveillance (of at-risk groups), and to provide and
disseminate information'' (3) on at-risk individuals to inform action.
An implementation of a NSPS in Scotland called ``Choose Live'' decreased
suicide rates by 20\% over 10 years. This sort of improvement in suicide
rates after implementing is implied in the 2018 report and lends to the
recommendation that national strategies should be implemented. This
topic is however nuanced, take the Figure 1 below for historical rates
suicide rates of different countries (4)(5)(6)(7). When we look
specifically at the Sweden and Switzerland's rates, they both fall in
\textasciitilde{}1972 and 1981 respectively. By comparing historical
events of each country some interesting hypotheses can be drawn.
Governmental changes coinciding in 1971 for both countries did not
dramatically improve the rate. Cultural events which affected how masses
might view a social issue had much greater temporal impacts on their
rates. For instance, Euthanasia in the 1980s Switzerland coincided with
an inversion of the curve of its rate. The cultural perception of
suicide as a ``bad'' thing became more accepted in certain circumstances
and may have translated to individual self-perceptions of suicidal
thoughts to be a more normal occurrence which would not be cause for
social abandonment. Sweden's ``Sexual Revolution'' translated to an
acceptance of a group, LGBT identifying individuals, which today has
been identified as an at-risk population. While government
implementation of NSPSs in time may lead to a reduction in suicide, it
may also be the cultural recognition of the issue, in addition to
specific policy actions which decrease overall suicide.

Countries still should consider establishing an authoritative agency,
tasked with the continued investigating, formulating, and implementing
of a National Suicide Prevention Strategy. It should follow actions like
those below from countries with success in reducing suicide (4): •
Reduce access to means and methods of suicide • View suicide as a
psychological mistake • Improve medical, psychological, and psychosocial
initiatives • Distribute knowledge about evidence-based methods for
reducing suicide • Raise skill levels among staff and other key
individuals in the care services • Perform ``root cause'' or event
analyses after suicide • Support voluntary organizations • Promote
public awareness campaigns highlighting the prevalence of suicide.
National strategies should not replace existing frameworks already in
place in local government either. By changing public perceptions,
reducing the stigmas associated with seeking help, and coming up with
national strategies to combat suicide, the rate of suicide can be
reduced.

\subsection{Alcohol Intake}\label{alcohol-intake}

Suicide is a complex societal problem with no singular cause. However,
harmful use of alcohol is among the major risk factors for suicide.
Policy makers should consider implementing measures designed to mitigate
the harmful use of alcohol as a means of reducing the rate of suicide.
According to the WHO, among the policy interventions that have proven
effective at reducing the harmful use of alcohol are varied. One is to
increase the price of alcohol via taxation, which is implemented
successfully in states such as Utah. Another is to enact and enforce
restrictions on alcohol advertising (across multiple types of media),
out of sight out of mind. And finally, enact and enforce restrictions on
the physical availability of retailed alcohol (via reduced hours of
sale), for example many ``dry states'' do not serve alcohol on Sundays.
(1) It is not recommended to remove access to alcohol completely as seen
in the disastrous US history lesion in the prohibition era. The
increased violence may not have been worth the decrease in suicide. (2)

\subsection{GDP Per Capita}\label{gdp-per-capita}

There is a negative correlation between GDP per capita adjusted for
Purchasing Power Parity (PPP) and suicide rates. While it is unknown why
this is, we believe that money should be spent to uncover more about the
relationship between income and suicide. Countries with lower GDPs tend
to have higher rates of suicide, which also tend to have lower quality
infrastructure, health care, and a plethora of other associated
industries. (1) With these lower quality services and access to them, at
risk individuals may have higher likelihood of suicide. An analysis on
income of specific income groups would shed more light as to whether low
income correlates to higher suicide or not. As such it is recommended
Invest in research to better understand potential relationships between
income instability, income protection and suicide at the individual
level. In addition, governments should pursue measures aimed at poverty
reduction and unemployment benefits to support economic well-being.

\section{Research Limitations}\label{research-limitations}

In any study there are limitations on what is considered in analysis. We
only considered a limited set of inputs and analysis measures in the
allotted time and would perform more had there been more. A breakdown of
the research limitations of scope, what was considered, and methodology,
how it was analyzed, are described below.

\subsection{Scope}\label{scope}

There were issues with some of our inputs, but when drilling down to
just the inputs used in the model, we can see room for improvement in
data quality. When we used GDP per Capita as a proxy for income, other
measures such as country-level median income should have been considered
in the future. This would have given a non-uniform distribution of
wealth in the country rather than a uniform distribution which is not
the case with income inequality. When measuring the liters of alcohol
consumed, we assumed a uniform consumption country-wide consumption
rate. This does not consider incidence of substance abuse. For Suicide
Policy (NSPS), the effectiveness of organizational response hard per
country is hard to gauge since local response vs federal not accounted
for in measurement. We did not consider local/cultural/interactional
measures making it difficult to make country-specific inferences in some
cases.

\subsection{Methodology}\label{methodology}

For our analysis we chose to use a country level scope, however this
cannot drill down to local or individual level, essentially limiting our
level of fidelity of reflecting on reality. For each country we only
used one year, as such our model assumes effects of each input are fixed
rather than temporally differing. When considering our inputs, we cannot
completely untangle the effect of variable interactions between another.
Higher level interactions and additional factors which may influence
suicide rates could be considered in the future. Finally, model
formulation limited our analysis strength. We chose to use multiple
linear regression for inferential and descriptive reasons, but more
complicated / non-linear relationships could be characterized better
with more complex approaches.

\newpage 

\section{Appendix}\label{appendix}

\subsection{Variable \& Data Sources}\label{variable-data-sources}

The following table describes the the full set of variables we
considered:

\begin{table}[H]
\centering 
\caption{Data Sources}
\
\begin{tabular}{p{3cm}p{7cm}p{5cm}}  
\hline  
  Input & Data Description  & Source  \\   
\hline 
Current Health Expenditure as a Percentage of GDP & This data provides an indication on the level of resources channeled to health relative to other uses. It shows the importance of the health sector in the whole economy and indicates the societal priority which health is given measured in monetary terms. & World Health Organization (2)  \\   
 \hline 
Labor force participation rate (female-male ratio) & Ratio of female to male of proportion of a country’s working-age population (ages 15 and older) that engages in the labor market, either by working or actively looking for work, expressed as a percentage of the working-age population. & United Nations Development Programme (1) \\   
\hline 
GDP per capita, PPP & Gross Domestic Product converted to international dollars using purchasing power parity (PPP) rates and divided by total population. This data is in terms of PPP in order to account for differences in the cost of living between countries. & World Bank (1) \\
\hline 
Liters of Alcohol per capita &  Total (sum of recorded and unrecorded alcohol) amount of alcohol consumed per person (15 years of age or older) over a calendar year, in liters of pure alcohol, adjusted for tourist consumption. & World Bank (2) \\
\hline 
Suicide Prevention Strategy &  Countries which are known have a stand-alone national suicide prevention strategy are included as 1s, else 0. Note that the plan must be stand-alone, and may not be integrated into another plan, in order to count in the dataset. & World Health Organization (3) \\
\hline
Psychiatrists in mental health, per 100,000 pop. & Number of Psychiatrists working in the mental health sector, per 100,000 population.  & World Health Organization (4)  \\
\hline
Mental hospitals, per 100,000 pop. & Number of hospitals dedicated to mental health per 100,000 population & World Health Organization (5) \\
\hline
\end{tabular} 
\end{table}

\newpage

\subsection{Defining Suicide Rate}\label{defining-suicide-rate}

In order to properly analyze and define a model, we must first define
suicide. The measure we utilized was age-standardized, meaning that it
is a weighted average of the age-specific mortality rates per 100,000
persons, where the weights are the proportions of persons in the
corresponding age groups of the WHO standard population (1). To
calculate the age standardized rate, denoted \emph{ASR} see Equation 1
below (2).

\emph{Equation 1: Age-Standardized Suicide Rate} \[
\begin{aligned}
& \text{ASR} = \frac{\sum{a_i w_i}}{\sum{w_i}} \ \ \text{for } i = 1,...,n,\ \text{where}\\
& a_i = \text{Age specific rate for group } i \\
& w_i = \text{The country standard population weight for group } i \\
& n = \text{The number of age groups considered}
\end{aligned}
\]

The age standardized rate was used instead of crude as it allows for an
age normalized view of suicide. In addition, there are no age-related
statistics in data inputs considered. Note that this measure was the key
outcome of interest for our model.

\subsection{Model Final Specification}\label{model-final-specification}

The following notation details the final specification of the model we
built. Note that we were able to gather data for \(n=166\) countries
around the world across all measures detailed below:

\[
\begin{aligned}
& Y_i = \beta_0 + \beta_1 x_{1i} + \beta_2 x_{2i} + \beta_3 x_{3i} + \beta_4 x_{4i}  + \epsilon_i,\ \text{where we assumed, } \epsilon_i \sim \mathbb{N}(0,\sigma_{Y}^2) \\
&\\
&\text{for } i = 1,...,n \ \text{country level measures, where} \\
& \\
& Y_i \ \ : \text{The estimated national suicide rate (per 100k population) for the} i^{\text{th}} \text{ country.(Box-cox transformed $\lambda = 0.4$)} \\
& x_{1i}\ : \text{The estimated national labor participation rate (percentage) for the } i^{\text{th}} \text{ country.}\\
& x_{2i}\ : \text{The log-transformed estimated per-person gross domestic product (GDP) (income) for the } i^{\text{th}} \text{ country.}\\
& x_{3i}\ : \text{An estimate of the national per-person average of liters of alcohol consumed annually for the } i^{\text{th}} \text{ country.}\\
& x_{4i}\ : \text{A binary indicator of the 'presence of a national suicide prevention strategy' in 2019 for the } i^{\text{th}} \text{ country.}\\
& \\
& \text{This yields fitted regression model: } \\
& \\
& \hat{Y_i} = \hat{\beta_0} + \hat{\beta_1} x_{1i} + \hat{\beta_2} x_{2i} + \hat{\beta_3} x_{3i} + \hat{\beta_4} x_{4i} \\
& \\
& \text{where, } \\
& \\
& \hat{\beta_0},\ \hat{\beta_1},\ \hat{\beta_2},\ \hat{\beta_3}, \ \text{and } \hat{\beta_4} \text{ were estimated by the method of iterative re-weighted least squares.} \\
\end{aligned}
\]

\subsection{Additional Details: Modeling
Approach}\label{additional-details-modeling-approach}

\subsubsection{Initial Model Choice and
Transformation}\label{initial-model-choice-and-transformation}

We developed a multiple linear regression model to infer properties
about how a handful of socioeconomic and cultural indicators impact
suicide rates. Initial Model, First, we analyzed the diagnostic plots to
understand if our model appropriately fits the data that we have. A
major red flag here was the Normal Q-Q plot, which shows if residuals
are normally distributed. The residuals deviate from the reference line
at the higher quintiles. In order to correct for this, our next step was
to try a Box Cox transformation on Y. Below is the log-likelihood plot
to determine the lambda value for the Box Cox transformation. A lambda
value of 0.4 was chosen.

We also applied log transformation to the variable GDP per capita to
better represent the relationship between this variable and the outcome
variable based on visual data exploration and resulting effect .

\subsubsection{Outlier Removal
Decisions}\label{outlier-removal-decisions}

We relied on quantitative measures of model leverage, and Q-Q plots to
identify extreme deviations from normality to identify and countries
that we removed from our analysis. However as we identified these
countries, we also compiled notes on specific qualititative
characteristics of these countries that may put their quantitative
outlier status into better context:

\emph{Barbados}: Caribbean's leading tourism island, transitioned from
agricultural to service based economy very successfully and has
\emph{very high human development} status in terms of the UNDP's human
development index in contrast with an extremely low suicide rate.

\emph{Guyana}: An extremely poor island country largely made up of
agricultural villages. It has very high alcohol and suicide statistics,
and the country's ministry of health identifies poverty, pervasive
stigma about mental illness, access to lethal chemicals, alcohol misuse,
interpersonal violence, family dysfunction and insufficient mental
health resources as key factors causing one of the highest suicide rates
in the world.

\emph{Japan}: Japan has a notoriously overworked and over stressed
population, although the country is very wealthy. Japan has a long
cultural history of considering certain types of suicides honorable, and
has relatively high cultural tolerance for suicide with a very high
suicide rate when compared to other rich nations.

\emph{Lesotho}: A small, landlocked, mountainous country in Africa with
the highest suicide rate in Africa, high levels of child labor, very
poor general health outcomes, and the second highest rate of
tuberculosis and HIV/AIDS in the world.

\subsubsection{Stepwise Variable Selection
Approach}\label{stepwise-variable-selection-approach}

We implemented a backwards stepwise algorithm based on AIC to remove
variables based on the AIC criterion, the results and variable inclusion
decisions are detailed below:

\begin{table}[H]
\centering 
\caption{Variable Selection Details: Backwards Stepwise Regression (AIC Criterion)}
\
\begin{tabular}{p{5cm}p{4cm}}  
\hline  
Variable & Model Inclusion Result \\  
\hline
GDP per capita, PPP & Included \\
\hline 
Prevalence of a national suicide prevention strategy & Included \\
\hline 
Liters of alcohol consumption per capita & Included \\
\hline 
Male to Female ratio of the labor participation rate & Included \\
\hline
Health Expenditure as a percentage of GDP & Removed \\
 \hline 
Psychiatrists working in mental health sector (per 100 000 population) & Removed \\   
\hline 
Mental hospitals (per 100 000 population) &  Removed \\
\hline 
\end{tabular} 
\end{table}

The final step in preparing the model was to implement the iteratively
weighted least squares algorithm to properly weight each instance in our
data. This was more effort to mitigate the effect of outliers on our
model. We performed 10 iterations of this algorithm, and final estimates
are provided in the next section.

\subsection{Model Summary Statistics}\label{model-summary-statistics}

\begin{table}[H] \centering 
  \caption {Regression Model Summary} 
  \label{tab:title} 
\begin{tabular}{@{\extracolsep{5pt}}lc} 
\\[-1.8ex]\hline 
\hline \\[-1.8ex] 
 & \multicolumn{1}{c}{\textit{Dependent variable:}} \\ 
\cline{2-2} 
\\[-1.8ex] & Suicide Rate (Box-Cox Transformed $\lambda = 0.4$) \\ 
\hline \\[-1.8ex] 
 Income (pp GDP) - Log Transformed & $-$0.404$^{***}$ \\ 
  & (0.080) \\ 
  & \\ 
Liters of Alcohol Consumed & 0.166$^{***}$ \\ 
  & (0.026) \\ 
  & \\ 
Suicide Prevention Strategy (Binary) & 0.562$^{***}$ \\ 
  & (0.185) \\ 
  & \\ 
Labor Participation Rate & 1.031$^{**}$ \\ 
  & (0.472) \\ 
  & \\ 
 Constant & 5.420$^{***}$ \\ 
  & (0.828) \\ 
  & \\ 
\hline \\[-1.8ex] 
Observations & 162 \\ 
R$^{2}$ & 0.412 \\ 
Adjusted R$^{2}$ & 0.397 \\ 
Residual Std. Error & 1.272 (df = 157) \\ 
F Statistic & 27.475$^{***}$ (df = 4; 157) \\ 
\hline 
\hline \\[-1.8ex] 
\textit{Note:}  & \multicolumn{1}{r}{$^{*}$p$<$0.1; $^{**}$p$<$0.05; $^{***}$p$<$0.01} \\ 
\end{tabular} 
\end{table}

\newpage 

\section{References}\label{references}

{[}1{]} \emph{Suicide: one person dies every 40 seconds}, 2019,
Retrieved from:
\url{https://www.who.int/news-room/detail/09-09-2019-suicide-one-person-dies-every-40-seconds},
Accessed: 2020-03-08

{[}2{]} \emph{Suicide: Key facts}, 2019, Retrieved from:
\url{https://www.who.int/news-room/fact-sheets/detail/suicide},
Accessed: 2020-03-08

{[}3{]} \emph{Suicide rate estimates, age-standardized estimates by
country}, 2019, Retrieved from:
\url{http://apps.who.int/gho/data/node.main.MHSUICIDEASDR?lang=en},
Accessed: 2020-03-08

{[}4{]} \emph{Alcohol-Related Risk of Suicidal Ideation, Suicide
Attempt, and Completed Suicide: A Meta-Analysis}, 2015, Retrieved from:
\url{https://www.ncbi.nlm.nih.gov/pmc/articles/PMC4439031/}, Accessed:
2020-04-05

{[}5{]} \emph{Does suicide always indicate a mental illness?}, 2009,
Retrieved from:
\url{https://www.ncbi.nlm.nih.gov/pmc/articles/PMC4222167/}, Accessed:
accessed 2020-04-12

{[}6{]} \emph{Suicide Prevention Framework}, 2016, Retrieved from:
\url{https://www.canada.ca/en/public-health/services/publications/healthy-living/suicide-prevention-framework.html},
Accessed: accessed 2020-04-12

{[}7{]} \emph{The Economic Decline of Zimbabwe}, 2009, Retrieved from:
\url{https://cupola.gettysburg.edu/cgi/viewcontent.cgi?article=1021\&context=ger},
Accessed: accessed 2020-04-12

{[}8{]} \emph{Alcohol Consumption}, 2018, Retrieved from:
\url{https://ourworldindata.org/alcohol-consumption}, Accessed: accessed
2020-04-06

{[}9{]} \emph{WHO MiNDbank}, 2020, Retrieved from:
\url{https://www.who.int/mental_health/mindbank/en/}, Accessed:
2020-04-16

{[}10{]} \emph{National suicide prevention strategies: Progress,
examples and indicators}, 2019, Retrieved from:
\url{https://apps.who.int/iris/handle/10665/279765}, Accessed: accessed
2020-04-17

{[}11{]} \emph{The Effects of War and Alcohol Consumption Patterns on
Suicide: United States, 1910-1933}, 1989, Retrieved from:
\url{https://apps.who.int/iris/rest/bitstreams/1174021/retrieve},
Accessed: 2020-04-17

{[}12{]} \emph{Global status report on alcohol and health 2018}, 2018,
Retrieved from:
\url{https://www.who.int/substance_abuse/publications/global_alcohol_report/en/},
Accessed: 2020-04-05

{[}13{]} \emph{Current health expenditure (CHE) as percentage of gross
domestic product (GDP) (\%)}, 2020, Retrieved from:
\url{https://www.who.int/data/gho/data/indicators/indicator-details/GHO/current-health-expenditure-(che)-as-percentage-of-gross-domestic-product-(gdp)-(-)},
Accessed: 2020-04-16

{[}14{]} \emph{Labor force participation rate (female-male ratio)},
2013, Retrieved from:
\url{http://hdr.undp.org/en/content/labour-force-participation-rate-female-male-ratio},
Accessed: 2020-04-16

{[}15{]} \emph{GDP per capita, PPP}, 2018, Retrieved from:
\url{https://data.worldbank.org/indicator/NY.GDP.PCAP.PP.CD}, Accessed:
2020-04-17

{[}16{]} \emph{Total alcohol consumption per capita}, 2016, Retrieved
from: \url{https://data.worldbank.org/indicator/SH.ALC.PCAP.LI},
Accessed: 2020-04-17

{[}17{]} \emph{Human Resources Data by country}, 2019, Retrieved from:
\url{https://apps.who.int/gho/data/node.main.MHHR?lang=en}, Accessed:
2020-04-17

{[}18{]} \emph{Facilities Data by country}, 2019, Retrieved from:
\url{https://apps.who.int/gho/data/node.main.MHFAC?lang=en}, Accessed:
2020-04-17

\end{document}
